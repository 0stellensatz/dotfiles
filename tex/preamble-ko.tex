% engine recommendation : xelatex
% class recommendation  : oblivoir
% note                  : use \usepackage{fapapersize}
%                             \usefapapersize{a,b,c,d,e,f}
%                         for customizing layout.

% packages

\usepackage{graphicx}

\usepackage{amsmath,amssymb,amsthm}
\usepackage{mathrsfs}
\usepackage{mathtools}
\usepackage{tikz}
\usepackage{tikz-cd}
\usepackage{mdframed}
\usepackage{physics}
\usepackage{halloweenmath}

\usepackage{hyperref}
\usepackage{fancyhdr}
\usepackage{setspace}

\usepackage[utf8]{inputenc}

\usepackage[hangul]{kotex}

% terminology

\newcommand{\termTheorem}{정리\rmfamily}
\newcommand{\termAxiom}{공리\rmfamily}
\newcommand{\termProposition}{명제\rmfamily}
\newcommand{\termLemma}{보조정리\rmfamily}
\newcommand{\termConjecture}{추측\rmfamily}
\newcommand{\termCorollary}{따름정리\rmfamily}
\newcommand{\termDefinition}{정의\rmfamily}
\newcommand{\termProblem}{문제\rmfamily}
\newcommand{\termSolution}{풀이\rmfamily}
\newcommand{\termExample}{예제\rmfamily}
\newcommand{\termRemark}{\normalfont 주\rmfamily}
\newcommand{\termNote}{\normalfont 노트\rmfamily}
\newcommand{\termProof}{\normalfont 증명\rmfamily}

% mathematics

\DeclarePairedDelimiter{\ceil}{\lceil}{\rceil}
\DeclarePairedDelimiter{\floor}{\lfloor}{\rfloor}

\newcommand{\SetR}{\mathbb{R}}
\newcommand{\SetQ}{\mathbb{Q}}
\newcommand{\SetZ}{\mathbb{Z}}
\newcommand{\SetN}{\mathbb{N}}
\newcommand{\SetC}{\mathbb{C}}
\newcommand{\SetF}{\mathbb{F}}

\newtheoremstyle{cjkplain}% name
{2pt}% space above
{10pt}% space below
{\upshape}% body font
{}% indent amount
{\bfseries}% theorem head font
{.}% punctuation after theorem head
{.5em}% space after theorem head
{}% theorem head spec (can be left empty, meaning `normal')

\theoremstyle{cjkplain}

\newtheorem{theorem}{\termTheorem}[section]
\newtheorem{axiom}{\termAxiom}[section]
\newtheorem{conjecture}{\termConjecture}[section]
\newtheorem{lemma}[theorem]{\termLemma}
\newtheorem{proposition}[theorem]{\termProposition}
\newtheorem{problem}{\termProblem}[section]
\newtheorem{corollary}{\termCorollary}[theorem]

\theoremstyle{definition}

\newtheorem{definition}{\termDefinition}[section]
\newtheorem{solution}{\termSolution}[problem]
\newtheorem*{example}{\termExample}

\theoremstyle{remark}

\newtheorem*{note}{\termNote}
\newtheorem*{remark}{\termRemark}

\renewcommand\qedsymbol{$\square$}
