% engine recommendation : xelatex
% class recommendation  : article

% packages

\usepackage{graphicx} % images.

\usepackage{amsmath,amssymb,amsthm} % math symbols, theorem-like, etc.
\usepackage{mathrsfs} % Ralph Smith's Formal Script, \mathscr{}.
\usepackage{mathtools} % more math symbols, arrows, etc.
\usepackage{tikz} % TikZ ist kein Zeichenprogramm.
\usepackage{tikz-cd} % TikZ + commutative diagrams.
\usepackage{mdframed} % frames, boxes, etc.
\usepackage{physics} % calculus symbols and tools, etc.
\usepackage{halloweenmath} % halloween + math!
\usepackage{xcolor}

\usepackage{enumitem} % replaces old enumerate.
\renewcommand{\labelenumi}{(\theenumi)}

\usepackage{hyperref} % hyperlinks, navigations.
\usepackage{fancyhdr} % fancy headers and footers.
\usepackage{lastpage}
\usepackage{setspace} % line spacing.

\usepackage[utf8]{inputenc}
\usepackage[autolanguage]{numprint}
\usepackage[hangul]{kotex}

\xetexkofontregime{hangul}[
    alphs=latin,
    parens=latin,
    colons=prevfont,
    hyphens=latin,
    puncts=prevfont,
    cjksymbols=hangul,
]

% terminology

\newcommand{\termTheorem}{정리\rmfamily}
\newcommand{\termAxiom}{공리\rmfamily}
\newcommand{\termProposition}{명제\rmfamily}
\newcommand{\termLemma}{보조정리\rmfamily}
\newcommand{\termConjecture}{추측\rmfamily}
\newcommand{\termCorollary}{따름정리\rmfamily}
\newcommand{\termDefinition}{정의\rmfamily}
\newcommand{\termProblem}{문제\rmfamily}
\newcommand{\termSolution}{풀이\rmfamily}
\newcommand{\termExample}{예\rmfamily}
\newcommand{\termRemark}{\normalfont 주\rmfamily}
\newcommand{\termNote}{\normalfont 노트\rmfamily}
\newcommand{\termProof}{\normalfont 증명\rmfamily}

% mathematics

\DeclarePairedDelimiter{\ceil}{\lceil}{\rceil}
\DeclarePairedDelimiter{\floor}{\lfloor}{\rfloor}

\newcommand{\SetR}{\mathbb{R}}
\newcommand{\SetQ}{\mathbb{Q}}
\newcommand{\SetZ}{\mathbb{Z}}
\newcommand{\SetN}{\mathbb{N}}
\newcommand{\SetC}{\mathbb{C}}
\newcommand{\SetF}{\mathbb{F}}

\newcommand{\RR}{\mathbb{R}}
\newcommand{\QQ}{\mathbb{Q}}
\newcommand{\ZZ}{\mathbb{Z}}
\newcommand{\NN}{\mathbb{N}}
\newcommand{\CC}{\mathbb{C}}
\newcommand{\FF}{\mathbb{F}}

\newtheoremstyle{cjkplain}% name
{}% space above
{}% space below
{\upshape}% body font
{}% indent amount
{\bfseries}% theorem head font
{.}% punctuation after theorem head
{.5em}% space after theorem head
{}% theorem head spec (can be left empty, meaning `normal')

\makeatletter
\def\cleartheorem#1{%
    \expandafter\let\csname#1\endcsname\relax
    \expandafter\let\csname c@#1\endcsname\relax
}
\makeatother

\theoremstyle{cjkplain}

\newtheorem{theorem}{\termTheorem}[section]
\newtheorem{conjecture}[theorem]{\termConjecture}
\newtheorem{lemma}[theorem]{\termLemma}
\newtheorem{proposition}[theorem]{\termProposition}
\newtheorem{corollary}[theorem]{\termCorollary}
\newtheorem{problem}[theorem]{\termProblem}

\theoremstyle{definition}

\newtheorem{definition}[theorem]{\termDefinition}
\newtheorem{example}[theorem]{\termExample}

\theoremstyle{remark}

\newtheorem*{note}{\termNote}
\newtheorem*{remark}{\termRemark}
\newtheorem*{solution}{\termSolution}

\renewcommand\qedsymbol{$\square$}
