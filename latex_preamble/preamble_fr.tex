% engine recommendation : xelatex
% class recommendation  : oblivoir
% note                  : use \usepackage{fapapersize}
%                             \usefapapersize{a,b,c,d,e,f}
%                         for customizing layout.

% packages

\usepackage{graphicx}

\usepackage{amsmath,amssymb,amsthm}
\usepackage{mathrsfs}
\usepackage{mathtools}
\usepackage{tikz}
\usepackage{tikz-cd}

\usepackage{url}
\usepackage{fancyhdr}
\usepackage{setspace}

\usepackage[utf8]{inputenc}
\usepackage[french]{babel}
\usepackage[autolanguage]{numprint}

\usepackage{kotex}

% terminology

\newcommand{\termTheorem}{Théorème}
\newcommand{\termAxiom}{Axiome}
\newcommand{\termProposition}{Proposition}
\newcommand{\termLemma}{Lemme}
\newcommand{\termConjecture}{Conjecture}
\newcommand{\termCorollary}{Corollaire}
\newcommand{\termDefinition}{Définition}
\newcommand{\termProblem}{Problème}
\newcommand{\termSolution}{Solution}
\newcommand{\termExample}{Exemple}
\newcommand{\termRemark}{Remarque}
\newcommand{\termNote}{Commentaire}
\newcommand{\termProof}{Preuve}

% mathematics

\DeclareMathOperator{\csch}{csch}
\DeclareMathOperator{\sech}{sech}
\DeclareMathOperator{\Li}{Li}
\DeclareMathOperator{\Imaginary}{\mathfrak{I}}
\DeclareMathOperator{\Real}{\mathfrak{R}}
\DeclareMathOperator{\GenLin}{GL}
\DeclareMathOperator{\SpecLin}{SL}
\DeclareMathOperator{\SpecOrth}{SO}
\DeclareMathOperator{\Kernel}{\text{Ker}}
\DeclareMathOperator{\Image}{\text{Im}}

\DeclarePairedDelimiter{\ceil}{\lceil}{\rceil}
\DeclarePairedDelimiter{\floor}{\lfloor}{\rfloor}

\newcommand{\SetR}{\mathbb{R}}
\newcommand{\SetQ}{\mathbb{Q}}
\newcommand{\SetZ}{\mathbb{Z}}
\newcommand{\SetN}{\mathbb{N}}
\newcommand{\SetC}{\mathbb{C}}
\newcommand{\SetF}{\mathbb{F}}

\theoremstyle{plain}

\newtheorem{theorem}{\termTheorem}[section]
\newtheorem{axiom}{\termAxiom}[section]

\newtheorem{lemma}[theorem]{\termLemma}
\newtheorem{conjecture}[theorem]{\termConjecture}
\newtheorem{proposition}[theorem]{\termProposition}

\newtheorem{corollary}{\termCorollary}[theorem]

\theoremstyle{definition}

\newtheorem{definition}{\termDefinition}[section]
\newtheorem{problem}{\termProblem}[section]

\newtheorem{solution}{\termSolution}[problem]

\newtheorem*{example}{\termExample}

\theoremstyle{remark}

\newtheorem*{note}{\termNote}
\newtheorem*{remark}{\termRemark}

\renewcommand\qedsymbol{$\square$}
